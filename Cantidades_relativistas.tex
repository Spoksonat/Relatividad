\documentclass{article}%
\usepackage[T1]{fontenc}%
\usepackage[utf8]{inputenc}%
\usepackage{lmodern}%
\usepackage{textcomp}%
\usepackage{lastpage}%
\usepackage{verbatim}
\usepackage{alltt}
\usepackage{listings}
\usepackage{amsmath}
\usepackage{amsfonts}
\usepackage{breqn}
\providecommand{\abs}[1]{\lvert#1\rvert}
\providecommand{\norm}[1]{\lVert#1\rVert}
\usepackage{graphicx}
\usepackage{float}
\usepackage{bbm}
\usepackage{physics}
\usepackage{slashed}
\usepackage{color}

\begin{document}%

\input{Titulop}

\section*{Tensor métrico}
\input{gp}

\section*{Símbolos de Christoffel}
\input{Christoffelp}

\section*{Componentes del tensor de Ricci}
\input{Riccip}

\section*{Componentes del tensor de Einstein}
\input{Einsteinp}

\section*{Tensor de Estres-Energía}
\input{estresp}

\section*{Ecuaciones de campo de Einstein}
\input{campop}

\section*{Determinante del tensor métrico}
\input{detgp}

\section*{Curvatura Gaussiana}
\input{curvaturap}

donde $R_{\alpha\beta\gamma\delta}$ es el tensor de Riemann.


\section*{Ecuaciones de la Geodésica}
\input{geo1p}
\input{geo2p}
\input{geo3p}
\input{geo4p}

\section*{Lagrangiano}
\input{lagrangianop}


\end{document}

